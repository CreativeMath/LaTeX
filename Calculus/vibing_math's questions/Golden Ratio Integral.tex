\documentclass{article}
\usepackage{geometry}
\geometry{a4paper, portrait, margin=1in}
\title{Golden Ratio Integral}
\author{Shreenabh Agrawal}
\begin{document}

\date{}
\renewcommand{\familydefault}{\rmdefault}
\maketitle{Question:}

$$\int_{1}^{\phi}\frac{\left(x^{2}+1\right)}{\left(x^{4}-x^{2}+1\right)}\ln\left(1+x-\frac{1}{x}\right)\,dx$$

\maketitle{Answer:}
Dividing numerator and denominator of the fraction by $x^{2},$

$$\int_{1}^{\phi}\frac{\left(1+\frac{1}{x^{2}}\right)}{\left(x^{2}-1\ +\ \frac{1}{x^{2}}\right)}\ln\left(x-\frac{1}{x}+1\right)dx$$

Now taking a substitution, $x-\frac{1}{x}$ as $t$,
so that, $$\left(1\ +\ \frac{1}{x^{2}}\right)dx=dt$$

Thus, the original integral (with new limits) can be rewritten as:
$$\int_{0}^{1}\frac{\ln\left(1+t\right)}{1+t^{2}}dx$$
Taking the trigonometric substitution $$ \tan\theta\ =\ t $$
such that $$\left(\sec^{2}\theta\ \right)d\theta\ =\ dt$$

The integral can be re-written with new limits as (let us denote it by (i))
$$\int_{0}^{\pi/4}\ln\left(1+\tan\theta\right)d\theta$$

By applying King's rule of sum of limits, the same integral can be re-written as
$$\int_{0}^{\pi/4}\ln\left(1+\tan\left(\frac{\pi}{4}-\theta\right)\right)d\theta$$

Applying expansion for tangent function,
$$\int_{0}^{\pi/4}\ln\left(1+\frac{1-\tan\theta}{1+\tan\theta}\right)d\theta $$

This can be simplified as (let us denote it by (ii)):
$$\int_{0}^{\pi/4}\left(\ln2\ -\ \ln\left(1+\tan\theta\right)\right)d\theta$$

Now adding (i) and (ii), we get twice of the value of the required definite integral as:
$$\int_{0}^{\pi/4}\ln2\ d\theta$$

Now substituting the limits and dividing by 2 (because we added two integrals in last step), we get our required answer as:
$$\frac{\pi}{8}\ln2$$
The approximate value of which is:
0.272198261288
\end{document}
