\documentclass{article}
\usepackage[utf8]{inputenc}
\usepackage{graphicx}

\title{Vibing Math (Geometry Problem)}
\author{Shreenabh Agrawal }
\date{\today}
\usepackage{amsmath}
\usepackage{geometry}
\geometry{a4paper, portrait, margin=1in}
\usepackage{natbib}
\usepackage{graphicx}
\usepackage{amssymb}
\usepackage[makeroom]{cancel}


\begin{document}

\maketitle

\section{Question}
Consider an isosceles triangle $ABC$ in which $AB = BC = 10$ units. Let $P_1, P_2, P_3, ... P_{60}$ be $60$ points on $BC$. Then, $$\sum_{i=1}^{60} (AP_i^2 + P_iB\times P_iC) = ?$$	

\section{Solution}
By Stewart's Theorem, 
$$(AB^2\times P_iC) + (AC^2\times P_iB) = BC ( AP_i^2 + P_iB\times P_iC)$$
Now, $AB = 10$ and $AC = 10$ ,so above expression becomes, 
$$100 (P_iB + P_iC) = BC (AP_i^2 + P_iB\times P_iC )$$
But here, $P_iB + P_iC = BC$, hence the above equation becomes
$$100 \cancel{(P_iB + P_iC)} = \cancel{BC} (AP_i^2 + P_iB\times P_iC )$$
$$100 = (AP_i^2 + P_iB \times P_iC)$$
This is true for all $i$ in $P_i$. Hence, the answer is
$$\sum_{i=1}^{60} (AP_i^2 + P_iB\times P_iC) = 60 \times 100$$
$$\boxed{= 6,000}$$
\end{document}

